\documentclass{article}
\usepackage{graphicx} % Required for inserting images

\title{Kodavimo teorijos užduoties ataskaita}
\author{Julius Jauga}
\date{December 2024}

\begin{document}

\maketitle

\section{Introduction}
This document provides a detailed report on the coding theory task. The aim is to decode Reed-Muller codes and analyze their performance.

\section{Background}
Reed-Muller codes are a family of linear error-correcting codes used in communications and data storage. They are known for their simplicity and robustness.

\section{Methodology}
The decoding process involves several steps:
\begin{enumerate}
    \item Syndrome calculation
    \item Error pattern identification
    \item Error correction
\end{enumerate}

\section{Results}
The decoding algorithm was tested on various codewords. The results show that the algorithm can correct up to 3 errors in a codeword of length 15.

\section{Conclusion}
The Reed-Muller decoding algorithm is effective for correcting errors in transmitted data. Future work will focus on optimizing the algorithm for faster performance.

\end{document}
